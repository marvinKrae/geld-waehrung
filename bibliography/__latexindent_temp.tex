@article{Buchner2007,
abstract = {The European Union Emissions Trading Scheme (EU ETS) is the world's first large experiment with an emissions trading system for carbon dioxide (CO 2 ) and it is likely to be copied by others if there is to be a global regime for limiting greenhouse gas emissions. After providing a brief discussion of the origins of the EU ETS, its relation to the Kyoto Protocol, and its precedents in Europe and the U.S., this paper focuses on allowance allocation—the process of deciding who will receive the newly limited rights to emit CO 2 . We describe how allowances were allocated in the EU ETS, with particular emphasis on the issues and problems encountered, including the lack of readily available installation-level data, the participants in the process, the use of projections, the choices of Member States with respect to auctioning, benchmarking, and new entrant provisions, and the difficult issue of deciding to whom the expected shortage was to be allocated. Finally, we discuss the recently available data on 2005 emissions and what they indicate concerning over-allocation, trading patterns, and abatement. We conclude with some observations about the broader implications of the EU ETS, what seems to be unique about CO 2 , and the fact that non-economic considerations inform the allocation of allowances. },
author = {Ellerman, A Denny and Buchner, Barbara K},
doi = {10.1093/reep/rem003},
issn = {1750-6816},
journal = {Review of Environmental Economics and Policy},
mendeley-groups = {GeldWaehrung},
number = {1},
pages = {66--87},
title = {{The European Union Emissions Trading Scheme: Origins, Allocation, and Early Results}},
url = {https://doi.org/10.1093/reep/rem003},
volume = {1},
year = {2007}
}
@misc{Umweltbundesamt,
author = {Umweltbundesamt},
title = {{Der Europ{\"{a}}ische Emissionshandel}},
url = {https://www.umweltbundesamt.de/daten/klima/der-europaeische-emissionshandel{\#}teilnehmer-prinzip-und-umsetzung-des-europaischen-emissionshandels},
urldate = {2020-04-07}
}
@misc{Emissionshandelsstelle2015,
author = {{Deutsche Emissionshandelsstelle}},
title = {{Emissionshandel in Zahlen}},
year = {2015}
}
@misc{Komission,
author = {{Europ{\"{a}}ische Komission}},
title = {{EU Emissions Trading System (EU ETS) | Climate Action}},
url = {https://ec.europa.eu/clima/policies/ets{\_}en},
urldate = {2020-04-07},
year = {{o. J.}}
}
@misc{FinanzenNet,
author = {Finanzen.net},
title = {{Langfristiger CO2 European Emission Allowancespreischart in Euro | CO2 European Emission Allowancespreis Tendenz}},
url = {https://www.finanzen.net/rohstoffe/co2-emissionsrechte/chart},
urldate = {2020-04-07},
year = {2020}
}
@misc{Netztransparenz2020,
author = {Netztransparenz},
title = {{EEG-Umlagen-{\"{U}}bersicht}},
url = {https://www.netztransparenz.de/EEG/EEG-Umlagen-Uebersicht},
urldate = {2020-04-07},
year = {2020}
}
@misc{Bundesnetzagentur2020,
author = {Bundesnetzagentur},
title = {{Preise und Rechnungen - EEG-Umlage Was ist die EEG-Umlage und wie funktioniert sie?}},
url = {https://www.bundesnetzagentur.de/SharedDocs/FAQs/DE/Sachgebiete/Energie/Verbraucher/Energielexikon/EEGUmlage.html},
urldate = {2020-04-07},
year = {2020}
}
@article{Bundesverfassungsgericht2016,
author = {Bundesverfassungsgericht},
month = {sep},
publisher = {Bundesverfassungsgericht},
title = {{1 BvR 1299/15}},
year = {2016}
}
@misc{Matthes2017,
author = {Matthes, Felix},
month = {feb},
title = {{EU-Umweltminister: Europa l{\"{a}}sst den Emissionshandel scheitern}},
url = {https://www.zeit.de/wirtschaft/2017-02/eu-umweltminister-emissionshandel-barbara-hendricks-co2-ausstoss},
year = {2017}
}
@article{SachverstandigenratzurBegutachtungdergesamtwirtschaftlichenEntwicklung,
author = {{Sachverst{\"{a}}ndigenrat zur Begutachtung der gesamtwirtschaftlichen Entwicklung}},
title = {{Jahresgutachten 2009/10 – Sechstes Kapital}},
year = {2010}
}



